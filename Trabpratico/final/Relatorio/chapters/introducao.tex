\chapter{Introdução}
\label{cap:intro}

O presente relatório documenta o trabalho prático referente a Unidade Curricular de Programação Orientada aos Objectos pertencente ao plano de estudos do 2º ano do Mestrado Integrado em Engenharia Informática.
Ao longo do semestre aprendemos a lidar com a linguagem de Programação orientada a objetos (também conhecida pela sua sigla \textit{POO}.) que é um modelo de análise, projeto e programação de sistemas de software baseado na composição e interação entre diversas unidades de software chamadas de objetos.  
Neste projeto foi-nos proposto o desenvolvimento de um serviço de transporte de passageiros \textit{UMeR}, utilizando para tal a linguagem \textit{JAVA} que é orientada a objetos.
Em termos gerais, pretendemos desenvolver uma aplicação em \textit{JAVA}, que é uma linguagem diferente das linguagens de programação convencionais, que são compiladas para código nativo, a linguagem \textit{JAVA} é compilada para um bytecode que é interpretado por uma máquina virtual (Java Virtual Machine, mais conhecida pela sua abreviação \textit{JVM}).
É pretendido que a aplicação permita que um utilizador consiga realizar uma viagem num dos taxis da \textit{UMeR} e que de certa forma consiga guardar toda a informação útil desde o registo de cada utilizador até aos extratos de viagens num determinado período, por exemplo. Ainda de realçar, é suposto conseguirmos os registos dos preços das viagens, assim como, marcação de viagens.
Tudo isto, iremos mostrar ao longo do relatório de forma clara e objetiva.


