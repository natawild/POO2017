\chapter{Arquitetura das Classes}
Neste capítulo falaremos do esqueleto da  aplicação UMeR, serão abordadas as classes presentes na aplicação assim como os atributos e funcionamento de cada uma e também as decisões tomadas.
\begin{figure}[htb]
	\centering
	\includegraphics[scale=0.45]{imagem/esquemaClasses}
	\caption{Esquema de Classes do BlueJ }
	\label{p2:fig:p2_classes}
\end{figure}

\newpage

\section{AtorInterface}
A interface AtorInterface servirá
 

\subsection{ Ator}
A classe  \textit{Ator} é uma classe abstrata e sevirá  como “modelo” para outras classes que dela herdem, não podendo ser instanciada por si só. Para ter um objeto de uma classe abstrata é necessário criar uma classe mais especializada que herda dela e então instanciar essa nova classe. Neste caso foram criadas as classes \textit{Admin}, \textit{Cliente} e  \textit{Motorista} que herdam a informação que está na classe superior (Ator).

\begin{figure}[htpb]
	\centering
	\includegraphics[scale=0.6]{imagem/atores}
	\caption{Classes }
	\label{p2:fig:p2_atoresr}
\end{figure}

As variáveis de instância da classe abstrata \textit{Ator} são apresentadas de seguida: 
\begin{verbatim}
private String email; 
private String nome; 
private String password; 
private String morada; 
private LocalDate dataNascimento; 
\end{verbatim}


\subsection{Admin}

Na classe \textit{Admin} estarão todos os dados herdados da classe \textit{Ator}. 


\subsection{Cliente}
Na classe \textit{Cliente} estarão todos os dados herdados da classe \textit{Ator}. 

\begin{verbatim}
private Coordenadas loc; //localização atual do cliente
private boolean emViagem;
\end{verbatim}
\subsection{Motorista}
Na classe \textit{Motorista} estarão todos os dados herdados da classe \textit{Ator}. 

\begin{verbatim}
private int grauCumprimentoHorario; //0-100
private int classificacao; //0-100
private double totalKms; 
private boolean disponivel;//verifica se está disponivel ou não 
private boolean horarioTrabalho; //verificar se está no horário de trabalho
private double destreza; //valor entre 0.5 e 1.9
private VeiculoInterface veiculo; 
private Historico viagemEmProcesso;
private int totalViagens;
\end{verbatim}

Decidimos que a destreza do motorista seria atribuida através da invocação de um random() que gera valores entre 0,5  e 1,9 afim de gerar alguma aleatoriedade nos tempos obtidos das viagens efetuadas. 
\begin{verbatim}
this.destreza = Utils.generateRandom(0.5f, 1.9f); 
\end{verbatim}

\newpage
\section{Veiculo}

\begin{figure}[htpb]
	\centering
	\includegraphics[scale=0.6]{imagem/veiculo}
	\caption{Classes }
	\label{p2:fig:p2_veiculos}
\end{figure}

\begin{verbatim}
private String matricula; 
private String marca; //Acrescentou-se a variável de instância marca, 
para o cliente poder escolher um carro com base na marca do veiculo

private float fiabilidade;//0 a 2 randon()
private Coordenadas loc;
\end{verbatim}

\section{Moto}
\begin{verbatim}
private static final int lugaresLivres = 1;
private static final double vm = 40.5; 
private static final double precoPorKm = 2.1;
\end{verbatim}

\subsection{MotoFilaEspera}
\begin{verbatim}
private List<Cliente> filaClientes;
\end{verbatim}

\section{CarroLig}

\begin{verbatim}
 private static final int lugaresLivres = 4;
private static final double vm = 65; 
private static final double precoPorKm = 3.5;
\end{verbatim}

\subsection{CarroFilaEspera}
\begin{verbatim}
private List<Cliente> filaClientes;
\end{verbatim}

\section{Carrinha}

\begin{verbatim}
private static int lugaresLivres = 8;
private static final double vm = 55;
private static final double precoPorKm = 5.1;
\end{verbatim}

\subsection{CarrinhaFilaEspera}
\begin{verbatim}
private List<Cliente> filaClientes;
\end{verbatim}

\newpage
\section{Historico}
\begin{figure}[htpb]
	\centering
	\includegraphics[scale=0.6]{imagem/historico}
	\caption{Classes }
	\label{p2:fig:p2_historico}
\end{figure}
\begin{verbatim}
private String emailCliente; 
private String emailMotorista; 
\end{verbatim}

\section{Utils}
Adicionalmente a classe \textit{Utils} tem implementado um método que encripta a password. E um método que gera números random com intervalos de 0.1. 

\subsection{Meteorologia}
\begin{verbatim}
public static final String sol = "Sol"; 
public static final String nevoeiro  = "Nevoeiro"; 
public static final String granizo = "Granizo";
public static final String chuva = "Chuva";
public static final String neve = "Neve"; 
\end{verbatim}

\subsection{Trânsito}
\begin{verbatim}
public static final String st = "Sem Transito"; 
public static final String tn = "Transito Normal"; 
public static final String mt  = "Muito Transito"; 
\end{verbatim}

\section{Coordenadas}
\begin{verbatim}
private double x;
private double y;
\end{verbatim}

As cordenadas também são iniciadas com o método random(). 
\begin{verbatim}
this.x=Utils.generateRandom(0f, 100f);
this.y=Utils.generateRandom(0f, 100f);
\end{verbatim}

O método getDistancia() calcula a distancia euclidiana, este será um método importante na execução da simulação de uma viagem. 
\begin{verbatim}
public double getDistancia (Coordenadas c){
    double distancia=0; 
    distancia = Math.sqrt( Math.pow((this.x - c.getX()),2 ) +
                           Math.pow((this.y - c.getY()),2 ));
    return distancia; 
}
\end{verbatim}

\section{BD}
\begin{verbatim}
private Map<String, AtorInterface> clientes;
private Map<String, AtorInterface> motoristas; 
private Map<String, AtorInterface> admins; 
private Map<String,VeiculoInterface> veiculos; 
private Set<Historico> historico;
\end{verbatim}

\section{UMeRMenu}
\begin{verbatim}
private String titulo;
private List<String> opcoes;
private int op;
\end{verbatim}

\section{UMeR}
\begin{verbatim}
private BDInterface baseDeDados;
private AtorInterface atorLoggado;
private int tentativasDeLoginFalhadas;
private Map<String, AtorInterface> atores;
\end{verbatim}

\section{UMeRApp}
\begin{verbatim}
private static UMeR umer;
private static UMeRMenu menu_principal;
private static UMeRMenu menu_registar_atores;
private static UMeRMenu menu_motorista;
private static UMeRMenu menu_cliente;
private static UMeRMenu menu_dados_pessoais;
private static UMeRMenu menu_cliente_efetuarViagem;
private static UMeRMenu menu_admin;
private static UMeRMenu menu_registar_veiculos;
private static UMeRMenu menu_solicitarViagem; 
private static UMeRMenu menu_inserir_coord_destino;
private static UMeRMenu menu_terminar_viagem; 
private static UMeRMenu menu_terminar_horario_trabalho;
private static UMeRMenu menu_iniciar_horario_trabalho;
private static UMeRMenu menu_proposta_viagem;
private static UMeRMenu menu_historico;
\end{verbatim}




